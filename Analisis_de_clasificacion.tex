%%%%%%%%%%%%%%%%%%%%%%%%%%%%%%%%%%%%%%%%%
% Beamer Presentation
% LaTeX Template
% Version 1.0 (10/11/12)
%
% This template has been downloaded from:
% http://www.LaTeXTemplates.com
%
% License:
% CC BY-NC-SA 3.0 (http://creativecommons.org/licenses/by-nc-sa/3.0/)
%
%%%%%%%%%%%%%%%%%%%%%%%%%%%%%%%%%%%%%%%%%

%----------------------------------------------------------------------------------------
%	PACKAGES AND THEMES
%----------------------------------------------------------------------------------------

\documentclass[usenames,dvipsnames,9pt]{beamer}

\mode<presentation> {

% The Beamer class comes with a number of default slide themes
% which change the colors and layouts of slides. Below this is a list
% of all the themes, uncomment each in turn to see what they look like.

%\usetheme{default}
%\usetheme{AnnArbor}
%\usetheme{Antibes}
%\usetheme{Bergen}
%\usetheme{Berkeley}
%\usetheme{Berlin}
%\usetheme{Boadilla}
%\usetheme{CambridgeUS}
%\usetheme{Copenhagen}
%\usetheme{Darmstadt}
%\usetheme{Dresden}
%\usetheme{Frankfurt}
%\usetheme{Goettingen}
%\usetheme{Hannover}
%\usetheme{Ilmenau}
%\usetheme{JuanLesPins}
%\usetheme{Luebeck}
\usetheme{Madrid}
%\usetheme{Malmoe}
%\usetheme{Marburg}
%\usetheme{Montpellier}
%\usetheme{PaloAlto}
%\usetheme{Pittsburgh}
%\usetheme{Rochester}
%\usetheme{Singapore}
%\usetheme{Szeged}
%\usetheme{Warsaw}

% As well as themes, the Beamer class has a number of color themes
% for any slide theme. Uncomment each of these in turn to see how it
% changes the colors of your current slide theme.

%\usecolortheme{albatross}
%\usecolortheme{beaver}
%\usecolortheme{beetle}
%\usecolortheme{crane}
%\usecolortheme{dolphin}
%\usecolortheme{dove}
%\usecolortheme{fly}
%\usecolortheme{lily}
%\usecolortheme{orchid}
%\usecolortheme{rose}
%\usecolortheme{seagull}
%\usecolortheme{seahorse}
%\usecolortheme{whale}
%\usecolortheme{wolverine}

%\setbeamertemplate{footline} % To remove the footer line in all slides uncomment this line
%\setbeamertemplate{footline}[page number] % To replace the footer line in all slides with a simple slide count uncomment this line

%\setbeamertemplate{navigation symbols}{} % To remove the navigation symbols from the bottom of all slides uncomment this line
}

\usepackage{graphicx} % Allows including images
\usepackage{booktabs} % Allows the use of \toprule, \midrule and \bottomrule in tables
\usepackage{amsfonts}
\usepackage{amsmath}
\usepackage{amssymb}
\usepackage{color, colortbl}
\usepackage{flafter}
\usepackage{enumerate}
\usepackage[T1]{fontenc}
\usepackage[latin1]{inputenc}
\usepackage{amsthm,latexsym,array}
\usepackage{float}
\usepackage{geometry}
\usepackage{mathtools}
\usepackage{multicol}
\usepackage{tcolorbox}
\usepackage{rotating}
\usepackage{ragged2e}
\usepackage{setspace}
\usepackage{subfigure}


\usefonttheme[onlymath]{serif}
\newcommand{\btVFill}{\vskip0pt plus 1filll}
%
\def\e#1{{\rm e}^{#1}}
\def\exp#1{{\rm exp}{#1}}
\def\frac#1#2{{{#1}\over{#2}}}
\def\binom#1#2{{{#1}\choose{#2}}}
\def\spot{$\bullet$\hspace{0.1cm}}
\def\le{\left}
\def\ri{\right}
\def\pro{\propto}
\def\prop{\propto}
%
\DeclareMathOperator*{\tr}{tr}
\DeclareMathOperator*{\logit}{logit}
\DeclareMathOperator*{\argmax}{arg\,max}
\DeclareMathOperator*{\argmin}{arg\,min}
%
\newcommand\iid{\mathrel{\overset{\makebox[0pt]{\mbox{\normalfont\tiny\sffamily iid}}}{\sim}}}
\newcommand\simiid{\mathrel{\overset{\makebox[0pt]{\mbox{\normalfont\tiny\sffamily iid}}}{\sim}}}
\newcommand\simind{\mathrel{\overset{\makebox[0pt]{\mbox{\normalfont\tiny\sffamily ind}}}{\sim}}}
\newcommand\eqd{\mathrel{\overset{\makebox[0pt]{\mbox{\normalfont\tiny\sffamily d}}}{=}}}
\newcommand{\ind}[1]{\mathbb{I}\left\{ #1 \right\}}
\newcommand{\pr}[1]{\mathbb{P}\text{r}\left[#1\right]}
\newcommand{\expec}[1]{\mathbb{E}\left[#1\right]}
\newcommand{\expe}[1]{\mathbb{E}\left[#1\right]}
\newcommand{\var}[1]{\mathbb{V}\text{ar}\left[#1\right]}
\newcommand{\sd}[1]{\text{SD}\left[#1\right]}
\newcommand{\cov}[1]{\mathbb{C}\text{ov}\left[#1\right]}
\newcommand{\coefvar}[1]{\text{CV}\left[#1\right]}
\newcommand{\diag}[1]{\text{diag}\left[#1\right]}
\newcommand{\expo}[1]{\exp{ \left\{ #1 \right\}}}
\newcommand{\ex}[1]{\exp{ \left\{ #1 \right\}}}
\newcommand{\quo}[1]{\textquotedblleft#1\textquotedblright}
\newcommand{\comi}[1]{\textquotedblleft#1\textquotedblright}
\newcommand\floor[1]{\lfloor#1\rfloor}
\newcommand{\ra}{\sqrt}
\newcommand{\bs}{\boldsymbol}
\newcommand{\TP}{\text{TP}}
\newcommand{\TN}{\text{TN}}
\newcommand{\FP}{\text{FP}}
\newcommand{\FN}{\text{FN}}
%
\def\A{\mathbf{A}}\def\a{\mathbf{a}}\def\Av{\boldsymbol{A}}\def\av{\boldsymbol{a}}
\def\B{\mathbf{B}}\def\b{\mathbf{b}}\def\Bv{\boldsymbol{B}}\def\bv{\boldsymbol{b}}
\def\C{\mathbf{C}}\def\c{\mathbf{c}}\def\Cv{\boldsymbol{C}}\def\cv{\boldsymbol{c}}
\def\D{\mathbf{D}}\def\d{\mathbf{d}}\def\Dv{\boldsymbol{D}}\def\dv{\boldsymbol{d}}
\def\E{\mathbf{E}}\def\e{\mathbf{e}}\def\Ev{\boldsymbol{E}}\def\ev{\boldsymbol{e}}
\def\F{\mathbf{F}}\def\f{\mathbf{f}}\def\Fv{\boldsymbol{F}}\def\fv{\boldsymbol{f}}
\def\G{\mathbf{G}}\def\g{\mathbf{g}}\def\Gv{\boldsymbol{G}}\def\gv{\boldsymbol{g}}
\def\H{\mathbf{H}}\def\h{\mathbf{h}}\def\Hv{\boldsymbol{H}}\def\hv{\boldsymbol{h}}
\def\I{\mathbf{I}}\def\i{\mathbf{i}}\def\Iv{\boldsymbol{I}}\def\iv{\boldsymbol{i}}
\def\J{\mathbf{J}}\def\j{\mathbf{j}}\def\Jv{\boldsymbol{J}}\def\jv{\boldsymbol{j}}
\def\K{\mathbf{K}}\def\k{\mathbf{k}}\def\Kv{\boldsymbol{K}}\def\kv{\boldsymbol{k}}
\def\L{\mathbf{L}}\def\l{\mathbf{l}}\def\Lv{\boldsymbol{L}}\def\lv{\boldsymbol{l}}
\def\M{\mathbf{M}}\def\m{\mathbf{m}}\def\Mv{\boldsymbol{M}}\def\mv{\boldsymbol{m}}
\def\N{\mathbf{N}}\def\n{\mathbf{n}}\def\Nv{\boldsymbol{N}}\def\nv{\boldsymbol{n}}
\def\O{\mathbf{O}}\def\o{\mathbf{o}}\def\Ov{\boldsymbol{O}}\def\ov{\boldsymbol{o}}
\def\P{\mathbf{P}}\def\p{\mathbf{p}}\def\Pv{\boldsymbol{P}}\def\pv{\boldsymbol{p}}
\def\Q{\mathbf{Q}}\def\q{\mathbf{q}}\def\Qv{\boldsymbol{Q}}\def\qv{\boldsymbol{q}}
\def\R{\mathbf{R}}\def\r{\mathbf{r}}\def\Rv{\boldsymbol{R}}\def\rv{\boldsymbol{r}}
\def\S{\mathbf{S}}\def\s{\mathbf{s}}\def\Sv{\boldsymbol{S}}\def\sv{\boldsymbol{s}}
\def\T{\mathbf{T}}\def\t{\mathbf{t}}\def\Tv{\boldsymbol{T}}\def\tv{\boldsymbol{t}}
\def\U{\mathbf{U}}\def\u{\mathbf{u}}\def\Uv{\boldsymbol{U}}\def\uv{\boldsymbol{u}}
\def\V{\mathbf{V}}\def\v{\mathbf{v}}\def\Vv{\boldsymbol{V}}\def\vv{\boldsymbol{v}}
\def\W{\mathbf{W}}\def\w{\mathbf{w}}\def\Wv{\boldsymbol{W}}\def\wv{\boldsymbol{w}}
\def\X{\mathbf{X}}\def\x{\mathbf{x}}\def\Xv{\boldsymbol{X}}\def\xv{\boldsymbol{x}}
\def\Y{\mathbf{Y}}\def\y{\mathbf{y}}\def\Yv{\boldsymbol{Y}}\def\yv{\boldsymbol{y}}
\def\Z{\mathbf{Z}}\def\z{\mathbf{z}}\def\Zv{\boldsymbol{Z}}\def\zv{\boldsymbol{z}}
%
\def\al{\alpha}\def\alv{\boldsymbol{\alpha}}
\def\be{\beta}\def\bev{\boldsymbol{\beta}}
\def\ga{\gamma}\def\gav{\boldsymbol{\gamma}}
\def\de{\delta}\def\dev{\boldsymbol{\delta}}
\def\del{\delta}\def\delv{\boldsymbol{\delta}}
\def\eps{\epsilon}\def\epsv{\boldsymbol{\epsilon}}
\def\veps{\varepsilon}\def\vepsv{\boldsymbol{\varepsilon}}
\def\ze{\zeta}\def\zev{\boldsymbol{\zeta}}
\def\te{\theta}\def\tev{\boldsymbol{\theta}}
\def\vte{\vartheta}\def\vtev{\boldsymbol{\vartheta}}
\def\io{\iota}\def\iov{\boldsymbol{\iota}}
\def\ka{\kappa}\def\kav{\boldsymbol{\kappa}}
\def\la{\lambda}\def\lav{\boldsymbol{\lambda}}
\def\lam{\lambda}\def\lamv{\boldsymbol{\lambda}}
\def\vrho{\varrho}\def\vrhov{\boldsymbol{\varrho}}
\def\si{\sigma}\def\siv{\boldsymbol{\sigma}}
\def\sig{\sigma}\def\sigv{\boldsymbol{\sigma}}
\def\vsi{\varsigma}\def\vsiv{\boldsymbol{\varsigma}}
\def\ups{\upsilon}\def\upsv{\boldsymbol{\upsilon}}
\def\vphi{\varphi}\def\vphiv{\boldsymbol{\varphi}}
\def\om{\omega}\def\omv{\boldsymbol{\omega}}\def\omev{\boldsymbol{\omega}}
\def\ome{\omega}
\def\etav{\boldsymbol{\eta}}
\def\xiv{\boldsymbol{\xi}}
\def\piv{\boldsymbol{\pi}}
\def\psiv{\boldsymbol{\psi}}
\def\phiv{\boldsymbol{\phi}}
\def\muv{\boldsymbol{\mu}}
\def\tev{\boldsymbol{\theta}}

%
\def\Ga{\Gamma}\def\GA{\mathbf{\Gamma}}
\def\Gam{\Gamma}\def\GAM{\mathbf{\Gamma}}
\def\De{\Delta}\def\DE{\mathbf{\Delta}}
\def\Del{\Delta}\def\DEL{\mathbf{\Delta}}
\def\Te{\Theta}\def\TE{\mathbf{\Theta}}
\def\La{\Lambda}\def\LA{\mathbf{\Lambda}}
\def\Lam{\Lambda}\def\LAM{\mathbf{\Lambda}}
\def\XI{\mathbf{\Xi}}
\def\PI{\mathbf{\Pi}}
\def\Si{\Sigma}\def\SI{\mathbf{\Sigma}}
\def\Sig{\Sigma}\def\SIG{\mathbf{\Sigma}}
\def\Ups{\Upsilon}\def\UPS{\mathbf{\Upsilon}}
\def\PHI{\mathbf{\Phi}}
\def\PSI{\mathbf{\Psi}}
\def\Om{\Omega}\def\OM{\mathbf{\Omega}}
\def\Ome{\Omega}\def\OME{\mathbf{\Omega}}
%
\def\teh{\hat{\theta}}\def\tevh{\hat{\boldsymbol{\theta}}}
\def\Unif{\small{\mathsf{Unif}}}
\def\MN{\small{\mathsf{Mult}}}
\def\Cat{\small{\mathsf{Cat}}}
\def\Dir{\small{\mathsf{Dir}}}
\def\DP{\small{\mathsf{DP}}}
\def\Ber{\small{\mathsf{Ber}}}
\def\Bin{\small{\mathsf{Bin}}}
\def\BetaBin{\small{\mathsf{BetaBin}}}
\def\NegBin{\small{\mathsf{NegBin}}}
\def\Nor{\small{\mathsf{N}}}
\def\normal{\small{\mathsf{N}}}
\def\Bet{\small{\mathsf{Beta}}}
\def\bet{\small{\mathsf{Beta}}}
\def\Gamd{\small{\mathsf{Gam}}}
\def\IGamd{\small{\mathsf{IGam}}}
\def\IG{\small{\mathsf{IGam}}}
\def\hyphen{\text{\textendash}}
%
\def\data{\text{data}}
\def\rest{\text{rest}}
\def\mle{\text{mle}}
\def\bayes{\text{bayes}}
\def\rest{\text{rest}}
\def\DIC{\text{DIC}}
%
\def\zerov{\boldsymbol{0}}
\def\onev{\boldsymbol{1}}
\def\onen{\boldsymbol{1}_n}
%
\def\reals{\mathbb{R}}
%
\newtheorem{theo}{Theorem}
%
\definecolor{violet}{rgb}{0.56, 0.0, 1.0}
\definecolor{LightCyan}{rgb}{0.88,1,1}
\definecolor{Gray}{gray}{0.9}

%%% ----------------------------------------------------------------------
% FIGURES ----------------------------------------------------------------
%=-=-=-=-=-=-=-=-=-=-=-=-=-=-=-=-=-=-=-=-=-=-=-=-=-=-=-=-=-=-=-=-=-=-=-=-=
%\graphicspath{{C:/PROJECT/presentation/figs/}}
%=-=-=-=-=-=-=-=-=-=-=-=-=-=-=-=-=-=-=-=-=-=-=-=-=-=-=-=-=-=-=-=-=-=-=-=-=



%TITLE PAGE
\title[]{An�lisis de Clasificaci�n} % The short title appears at the bottom of every slide, the full title is only on the title page
\subtitle{ }
%\newline \textbf{JASA}, Vol. 91, No. 433 (Mar., \textbf{1996}), pp. 142--153}

\author[Juan Sosa]{\LARGE{Juan Sosa, PhD}} % Your name
\institute[Universidad Externado] % Your institution as it will appear on the bottom of every slide, may be shorthand to save space
{
%University of California, Santa Cruz \\ % Your institution for the title page
\begin{figure}[h!]
\centering
\includegraphics[scale=.15]{./figs/logo-UE.pdf}
\end{figure}
}
\date{I - 2018} % Date, can be changed to a custom date

%\logo{\includegraphics[height=1.4cm]{Grateful_Slug.png}}


%\AtBeginSection[]
%{
%  \begin{frame}<beamer>
%    \frametitle{Outline}
%    \tableofcontents[currentsection]
%  \end{frame}
%}



\begin{document}

\begin{frame}

\titlepage % Print the title page as the first slide

\end{frame}


%OVERVIEW
%\begin{frame}
%\frametitle{Overview} % Table of contents slide, comment this block out to remove it
%\tableofcontents % Throughout your presentation, if you choose to use \section{} and \subsection{} commands, these will automatically be printed on this slide as an overview of your presentation
%\end{frame}


%\section{ }


\begin{frame}{An�lisis de Clasificaci�n (Clustering)}

\begin{block}{Objetivo}
\centering
Identificar grupos (\textit{clusters}) de individuos que sean homog�neos dentro de los grupos y heterog�neos entre grupos.
\end{block}

\begin{block}{Observaciones}
\begin{itemize}
  \item Diferente del an�lisis discriminante!
  \item Los grupos no est�n definidos a priori (o si incluso existen grupos).
  \item Es necesario considerar una distancia.
  \item Cu�ntos grupos?
  \item Como asignar los individuos?
\end{itemize}
\end{block}

\begin{block}{M�todos}
\begin{itemize}
  \item Clasificaci�n jer�rquica.
  \item Agrupamiento de $K$-medias.
  \item Agrupamiento basado en el modelo (estructura probabil�stica).
\end{itemize}
\end{block}

\end{frame}




\begin{frame}{Agrupamiento jer�rquico}

\begin{block}{Algoritmo}
\begin{enumerate}
  \item \textit{Start:} $C_1,C_2,\ldots,C_n$ (singletones), i.e., $K=n$.
  \item Encontrar y unir el par de clusters m�s \textit{cercanos}, $C_i$ y $C_j$.
  \item Decrecer el numero de clusters en 1, i.e., $K \leftarrow K - 1$.
  \item \textit{Stop} si $K = 1$. De lo contrario, volver al paso 2.
\end{enumerate}
\end{block}


\begin{block}{Distancias (similaridad) entre individuos}
\begin{itemize}
  \item $d_{ij} = \sqrt{\sum_\ell (x_{i,\ell} - x_{j,\ell})^2}$ :\quad\, Euclidiana.
  \item $d_{ij} = \max_\ell |x_{i,\ell} - x_{j,\ell}|$ : \quad \,\,\,\,\,Norma m�xima.
  \item $d_{ij} = \sum_\ell |x_{i,\ell} - x_{j,\ell}|$ : \qquad\,\,\,\,Manhattan.
  \item $d_{ij} = \left( \sum_\ell (x_{i,\ell} - x_{j,\ell})^p \right)^{1/p}$ : Minkowski.
\end{itemize}
\end{block}

\begin{block}{Distancias entre clusters $A$ y $B$}
\begin{itemize}
  \item $d_{AB} = \min_{i\in A,j\in B} \{ d_{ij} \}$ : \quad\,\,\,\,\,\,\,\textit{single linkage clustering}.
  \item $d_{AB} = \max_{i\in A,j\in B} \{ d_{ij} \}$ : \quad\,\,\,\,\,\,\textit{complete linkage clustering}.
  \item $d_{AB} = \tfrac{1}{n_a\,n_B} \sum_{i\in A}\sum_{i\in B} d_{ij}$ :  \textit{group average clustering}.
\end{itemize}
\end{block}

\end{frame}



\begin{frame}{Agrupamiento jer�rquico (cont.)}

\begin{block}{C�mo elegir $K$?}
Examinar los tama�os de los cambios de altura en el dendrograma y tomar un
\quo{gran} cambio para indicar el n�mero apropiado de clusters para los datos.
\end{block}

\begin{block}{Cons: \textit{chaining}}
Tendencia juntar puntos intermedios a un cluster ya establecido en lugar de inicializar un nuevo cluster.
\end{block}

\begin{block}{Observaciones}
\begin{itemize}
  \item Se debe ser cuidadoso al elegir las distancias (conmesurabilidad).
  \item No tiene estructura probabil�stica.
\end{itemize}
\end{block}

\end{frame}




\begin{frame}{Agrupamiento de $K$-medias}

\begin{block}{M�todo}
Encontrar la partici�n de $n$ individuos en $K$ grupos que minimicen el \textit{within-group sum of squares} (WGSS):
$$
\text{WGSS} = \sum_{j=1}^p \sum_{\ell=1}^K \sum_{i\in G_\ell} (x_{ij} - \bar{x}_j^{(\ell)})^2
$$
donde $\bar{x}_j^{(\ell)} = \sum_{i\in G_\ell} x_{ij}$ es la media en el grupo $G_\ell$ con la variable $j$.
\end{block}

\begin{block}{Algoritmo (Steinley, 2008)}
\begin{itemize}
  \item Partici�n inicial (agrupamiento jer�rquico).
  \item Calcular el cambio en el criterio de agrupamiento moviendo cada individuo de su propio cluster a otro cluster.
  \item Ejecutar el cambio que conlleve al mayor cambio en el valor del criterio de agrupamiento.
  \item Repetir pasos 2. y 3. hasta que el movimiento de ning�n individuo cause una mejora.
\end{itemize}
\end{block}

\end{frame}



\begin{frame}{Agrupamiento de $K$-medias (cont.)}

\begin{block}{Propiedad}
\vspace{-10pt}
\begin{align*}
\text{SC TOTAL} &= {\color{blue} \text{SC ENTRE}}\quad \,\qquad +\quad {\color{red} \text{SC DENTRO}}\\
                &= {\color{blue} \text{SC EXPLICADA}}\quad +\quad {\color{red} \text{SC NO EXPLICADA}}
\end{align*}
\end{block}

\begin{block}{}
$$
\text{Porcentaje de variabilidad explicado} = \frac{\text{SC EXPLICADA}}{\text{SC TOTAL}}
$$
\end{block}

\begin{block}{C�mo elegir $K$?}
Graficar el WGSS frente a $K$ y elegir aquel valor de $K$ donde el decrecimiento del WGSS no sea \quo{significativo}.
\end{block}

\begin{block}{Cons}
\begin{itemize}
  \item No es invariante a la escala de medici�n (conmensurabilidad).
  \item Tiende a construir grupos con estructura \quo{esferica}.
\end{itemize}
\end{block}

\begin{block}{Observaciones}
\begin{itemize}
  \item No tiene estructura probabil�stica.
\end{itemize}
\end{block}

\end{frame}


\begin{frame}{Agrupamiento basado en el modelo}

\begin{block}{Objetivo}
Desarrollar un modelo estad�stico que caracterice el mecanismo aleatorio que explique como se genera la data.
\end{block}

\begin{block}{Modelo}
$$
f(\xv;\piv,\boldsymbol{\theta}) = \sum_{j=1}^K \pi_j f_j(\xv,\boldsymbol{\theta}_j)
$$
donde $\sum_{j=1}^J \pi_j = 1$ (probabilidades de la mezcla).
\end{block}

\begin{block}{Objetivo}
Crear los grupos a partir de las probabilidades a posteriori:
$$
\pr{\text{Grupo $k$}\mid \mathbb{X}} = \frac{\pi_k\,f(\xv\mid\muv_k,\SIG)}{\sum_{j=1}^K \pi_j\,f(\xv\mid\muv_j,\SIG)}
$$
\end{block}

\end{frame}




\begin{frame}{Agrupamiento basado en el modelo (cont.)}

\begin{block}{Diagn�sticos}
Establecer la estabilidad del agrupamiento:
\begin{itemize}
  \item Fuerza de predicci�n: Cross-Validation (OK si superior a 0.8 o 0.9).
  \item Indice de Jaccard: Jittering o Bootstraping (OK si superior a 0.75).
\end{itemize}
\end{block}

\end{frame}





\end{document} 